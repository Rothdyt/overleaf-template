\documentclass[10pt]{article}
\usepackage{yd}
\usepackage[yyyymmdd]{datetime}

\renewcommand{\dateseparator}{--}
\title{Homework 2}
\author{Yutong Dai}
\date{Last Update: \today}
\begin{document}
\maketitle


\section{Q1}

\section{Q2}
Prove of disprove the following

\begin{itemize}
\item [a)] $f(n)=O(g(n)) \Rightarrow g(n)=O(f(n))$
\item [b)] $f(n)+g(n)=\Theta(\max (f(n), g(n)))$
\item [c)] $f(n)=O\left((f(n))^{2}\right)$
\item [d)] $f(n)=O(g(n)) \Rightarrow g(n)=\Omega(f(n))$
\item [e)] $f(n)+o(f(n))=\Theta(f(n))$
\end{itemize}

\begin{proof}
\item 
\begin{itemize}
    \item [a)] This is \texttt{False}. Consider $f(n)=n$ and $g(n)=n^2$. It's clear that there doesn't exist such a $n_0>0$ and constant $c$ such that $cn\geq n^2$ for all $n\geq n_0$ by the fact that $\lim_{n
    \rightarrow} f(n)/g(n)=0$.
    
    \item [b)] This is \texttt{True}. Here we assume both $f(n)$ and $g(n)$ are positive for all $n$. Denote $h(n):=\max\{f(n), g(n)\}$. It's clear that 
    $$h(n) \leq f(n) + g(n) \leq 2 h(n) \text{ for all }n\geq 1.$$
    Then $f(n)+g(n)=\Theta(\max (f(n), g(n)))$.
    \item [c)] This is \texttt{False}. Consider $f(n)=\frac{1}{n}$ and $g(n)=(f(n))^2=\frac{1}{n^2}$. Then $\lim_{n
    \rightarrow} g(n)/f(n)=0$.
    \item [d)] This is \texttt{True}. By the definition of $\Omega$, $f(n)=\Omega(g(n)) \Leftrightarrow \limsup _{n \rightarrow \infty}\left|\frac{f(n)}{g(n)}\right|>0$. Since $f(n)=O(g(n))$, then $\limsup _{n \rightarrow \infty}\frac{g(n)}{f(n)}>0$, which indicates $g(n)=\Omega(f(n))$.
    \item [e)] This is \texttt{True}. Assume $f(n)$ is positive. Note that, for sufficiently large $n_0$,
    $$
        \frac{1}{2} f(n) \leq f(n) +o(f(n)) \leq 2f(n)
    $$
    holds for all $n>n_0$. By definition, $f(n)+o(f(n))=\Theta(f(n))$.
\end{itemize}
\end{proof}


% \bibliographystyle{apalike}
% \bibliography{ref}
\end{document}